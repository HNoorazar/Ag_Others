\documentclass[serif, xcolor={dvipsnames}]{beamer} % serif, mathserif


% \usepackage{fourier}
\usepackage{newunicodechar}
\newcommand\Warning{%
 \makebox[1.4em][c]{%
 \makebox[0pt][c]{\raisebox{.1em}{\small!}}%
 \makebox[0pt][c]{\color{red}\Large$\bigtriangleup$}}}%
\newunicodechar{⚠}{\Warning}

\usepackage{stackengine}
\usepackage{scalerel}
\newcommand\dangersign[1][2ex]{%
  \renewcommand\stacktype{L}%
  \scaleto{\stackon[1.3pt]{\color{red}$\triangle$}{\tiny !}}{#1}%
}



\usepackage{fontawesome} % lightbulb
% we can define where figures are located
\usepackage{enumerate}
\usepackage{etaremune} % reverse order for enumerate
\usepackage{hhline}
\usepackage{listings}
\newcommand{\idea}{{\faLightbulbO~}}
\newcommand{\Rq}{{\faSearch~}}

\newcommand{\xitem}{%
  \par\hangindent3em\hangafter0
  \noindent\llap{$\triangleright$\enspace}%
  \ignorespaces}
  
% \def\myitem{\hangindent=5em$\triangleright$}
\def\myitem{\hangindent=5em}

\usepackage[natbib=true,style=authoryear,backend=bibtex,useprefix=true,giveninits=true]{biblatex}
% \usepackage[backend=bibtex]{biblatex}
\DeclareNameAlias{author}{last-first}
\addbibresource{./beamer_Refs.bib}
\usepackage[capitalise]{cleveref}

\def\code#1{{\scriptsize\texttt{#1}}}
% hypersetup only changes the year in citation.
\hypersetup{
    colorlinks=true,
    linkcolor=white, % color1 : will be black
    filecolor=red,
    urlcolor=ForestGreen,
    citecolor=cyan,
    bookmarksopen=false,
    pdftitle={Title},
    pdfauthor={Author},
}

\DeclareCiteCommand{\cite}
  {\color{cyan}\usebibmacro{prenote}}%
  {\usebibmacro{citeindex}%
   \usebibmacro{cite}}
  {\multicitedelim}
  {\usebibmacro{postnote}
  }



\usepackage[scaled=.9]{helvet} % platino activation
\usepackage{tabularx,multirow}
\usepackage{colortbl}
\RequirePackage{booktabs}

\colorlet{shadecolor}{gray!40}

\usepackage{chngcntr}
\usepackage{tcolorbox}

\setbeamertemplate{footline}[text line]{}
\setbeamertemplate{navigation symbols}{}
\usepackage{subcaption}
\usepackage{amsmath}
\usepackage{amsthm}
\usepackage{amsfonts}
\usepackage{amssymb}
\usepackage{calrsfs}
\usepackage{multicol}
%\captionsetup{font={small,stretch=0.80}}
\DeclareCaptionLabelFormat{andtable}{#1~#2  \&  \tablename~\thetable}

\newtheorem{thm}{Theorem}[section] % "[section]" restarts the theorem counter at every new section.
\newtheorem{Coro}[thm]{Corollary}
\newtheorem{lem}[thm]{Lemma}
\newtheorem{rem}[thm]{Remark}
\newtheorem{exm}[thm]{Example}
\newtheorem{deff}{Definition}



\usepackage{textpos} % i donno what this does

% \usepackage{etoolbox}
% \usepackage{mathptmx}
\usepackage{graphicx}
\usecolortheme{beaver}

% \usetheme{Berlin}

\usetheme{Boadilla} % adds date and page number
% \usetheme{Copenhagen}
% \usetheme{Ilmenau}
% \setbeamertemplate{blocks}[rounded][shadow=false]
% \addtobeamertemplate{block begin}{\pgfsetfillopacity{0.8}}{\pgfsetfillopacity{1}}
\setbeamercolor*{structure}{fg=mygreen}
\setbeamercolor*{block title example}{fg=blue!50,bg= blue!10}
\setbeamercolor*{block body example}{fg= blue,bg= blue!5}

\usepackage{ulem}
\usepackage{cancel}
\usefonttheme{professionalfonts}
{\vspace*{.10mm}}
\usetheme[height=10mm]{Rochester}



\definecolor{mygreen}{cmyk}{0.82,0.11,1,0.25}
\definecolor{aliceblue}{rgb}{0.94, 0.97, 1.0}

\definecolor{codegreen}{rgb}{0,0.6,0}
\definecolor{codegray}{rgb}{0.5,0.5,0.5}
\definecolor{codepurple}{rgb}{0.58,0,0.82}
\definecolor{backcolour}{rgb}{0.95,0.95,0.92}

\definecolor{bittersweet}{rgb}{1.0, 0.44, 0.37}
\definecolor{Gray}{gray}{0.85}
% \definecolor{LCyan}{rgb}{0.88,1,1}
\definecolor{mgreen}{rgb}{0.0,0.5,0.0}
%% \definecolor{mgreen}{rgb}{0.2, 0.8, 1}
\definecolor{brickred}{rgb}{0.8, 0.25, 0.33}
\definecolor{bleudefrance}{rgb}{0.19, 0.55, 0.91}
\definecolor{AB}{rgb}{0.94, 0.97, 1.0}
\definecolor{AB}{rgb}{0.9,.81,.68}
\definecolor{azureWeb}{rgb}{0.94, 1.0, 1.0}
\definecolor{beaublue}{rgb}{0.74, 0.83, 0.9}
\definecolor{linen}{rgb}{0.98, 0.94, 0.9}
\definecolor{magnolia}{rgb}{0.97, 0.96, 1.0}
\definecolor{moccasin}{rgb}{0.98, 0.92, 0.84}
\definecolor{navajowhite}{rgb}{1.0, 0.87, 0.68}
\definecolor{palecornflowerblue}{rgb}{0.67, 0.8, 0.94}



\title{Making a Case}
\author{HN}
\institute[WSU]{Washington State University}
\date{\today}

\titlegraphic{\includegraphics[width=.8cm]{cat1}}
%%%%%%%%%%%%%%%%
%%%%%%%%%%%%%%%% Fonts
\usepackage[osf, sc]{mathpazo}
% \renewcommand{\familydefault}{\sfdefault}
% \usepackage{lmodern}
% \usepackage{tgschola} % western type font
% \renewcommand{\familydefault}{\sfdefault} % nice
%\usepackage[sfdefault]{carlito}

% the following breaks long titles in References 
\DeclareFieldFormat[book,article]{title}{\textit{#1}}
%%%%%%%%%%%%%%%%%%%%%%%%%%%%%%%%%%%%%%%%%%%%%%%%%%%%%%%%%
\begin{document}
\maketitle

%\addtobeamertemplate{frametitle}{}{%}
%%%%%%%%%%%%%%%%%%%%
%%%%%%%%%%%%%%%%%%%%
%%%%%%%%%%%%%%%%%%%%

%%%%%%%%%%%%%%%%%%%%
%%%%%%%%%%%%%%%%%%%%
%%%%%%%%%%%%%%%%%%%%

\begin{frame}
\frametitle{General}

\begin{itemize}
\item Anything worth making needs patience and practice
\item If you want mastery, you need to immerse yourself in it till it becomes your second nature, just like walking

\vspace{.5in}
\item Top(?) two take aways from graduate school
\begin{enumerate}
\item Try not to have a bias and be playful \Rq
\item When you are wrong, you are wrong \dangersign
\end{enumerate}
\end{itemize}
\end{frame}

%%%%%%%%%%%%%%%%%%%%%
%%%%%%%%%%%%%%%%%%%%%
%%%%%%%%%%%%%%%%%%%%%
\begin{frame}[t]
\frametitle{Source of This Notes}

\vspace{-.18in}
\begin{figure}[htbp]
    \centering
    \begin{minipage}{0.3\textwidth}
        \centering
        \includegraphics[width=\linewidth]{ISLP_cover}
    \end{minipage}%
    \hspace{.2in}
    \begin{minipage}{0.5\textwidth}
       Notes from\\ {\bf An Introduction to Statistical Learning} {\tiny with Applications in Python}
    \end{minipage}
\end{figure}

\begin{itemize}
\item It's an easy read and as the name suggests, it's just Introduction. Good for intuition building
\item Its PDF is available for free from the Authors' website.
\item Sign-up \& receive coupons (30\%-50\% off)
\end{itemize}

\end{frame}
%%%%%%%%%%%%%%%%%%%%%
%%%%%%%%%%%%%%%%%%%%%
%%%%%%%%%%%%%%%%%%%%%
\begin{frame}[t]
\frametitle{Last thing first (There is no magic wand)}

\begin{itemize}[<+->]
\item There is no unique tool that outperforms other methods (there is no free lunch) 
\item There is no tool that is both accurate (in predicting) AND interpretable (unless we get lucky?!)
\end{itemize}

\pause
\vspace{.05in}
\begin{tcolorbox}
\centering
{\bf define your goal and deal with it}
\end{tcolorbox}

\pause
\vspace{-.1in}
\begin{figure}[htbp]
\centering
\includegraphics[width=\linewidth]{model_flex_inter}
\end{figure}

\end{frame}

%%%%%%%%%%%%%%%%%%%%%%%%%%%%%
%%%%%%%%%%%%%%%%%%%%%%%%%%%%%
%%%%%%%%%%%%%%%%%%%%%%%%%%%%%

%%%%%%%%%%%%%%%%%%%%%%%%%%%%%
%%%%%%%%%%%%%%%%%%%%%%%%%%%%%
%%%%%%%%%%%%%%%%%%%%%%%%%%%%%


%%%%%%%%%%%%%%%%%%%%%%%%%%%%%
%%%%%%%%%%%%%%%%%%%%%%%%%%%%%
%%%%%%%%%%%%%%%%%%%%%%%%%%%%%

%%%%%%%%%%%%%%%%%%%%%%%%%%%%%
%%%%%%%%%%%%%%%%%%%%%%%%%%%%%
%%%%%%%%%%%%%%%%%%%%%%%%%%%%%

%%%%%%%%%%%%%%%%%%%%%%%%%%%%%
%%%%%%%%%%%%%%%%%%%%%%%%%%%%%
%%%%%%%%%%%%%%%%%%%%%%%%%%%%%

%%%%%%%%%%%%%%%%%%%%%%%%%%%%%
%%%%%%%%%%%%%%%%%%%%%%%%%%%%%
%%%%%%%%%%%%%%%%%%%%%%%%%%%%%

%%%%%%%%%%%%%%%%%%%%%%%%%%%%%
%%%%%%%%%%%%%%%%%%%%%%%%%%%%%
%%%%%%%%%%%%%%%%%%%%%%%%%%%%%

%%%%%%%%%%%%%%%%%%%%%%%%%%%%%
%%%%%%%%%%%%%%%%%%%%%%%%%%%%%
%%%%%%%%%%%%%%%%%%%%%%%%%%%%%

%%%%%%%%%%%%%%%%%%%%%%%%%%%%%
%%%%%%%%%%%%%%%%%%%%%%%%%%%%%
%%%%%%%%%%%%%%%%%%%%%%%%%%%%%

%%%%%%%%%%%%%%%%%%%%%%%%%%%%%
%%%%%%%%%%%%%%%%%%%%%%%%%%%%%
%%%%%%%%%%%%%%%%%%%%%%%%%%%%%
\begin{frame}[t]
\frametitle{More Resources}

\begin{itemize}
\item Guesstimation:~\cite{weinstein2008guesstimation}
\item First Course in Probability~\cite{ross1976first}
\item Introduction to Regression by Montgomery~\cite{montgomery2021introduction}
\end{itemize}

\end{frame}

%%%%%%%%%%%%%%%%%%%%%%%%%%%%%
%%%%%%%%%%%%%%%%%%%%%%%%%%%%%
%%%%%%%%%%%%%%%%%%%%%%%%%%%%%
\begin{frame}[allowframebreaks,t]{References} 
\frametitle{Some Definition}


\begin{deff}[Variance of a method] Variance refers to the amount by which
$\hat f$ would change if we estimated it using a different training data set.
\end{deff}
\pause
\begin{deff}[Bias of a method] Bias refers to the error that is introduced by approximating a real-life problem, which may be extremely complicated, by a much simpler model.
\end{deff}

\end{frame}

%%%%%%%%%%%%%%%%%%%%%%%%%%%%%
%%%%%%%%%%%%%%%%%%%%%%%%%%%%%
%%%%%%%%%%%%%%%%%%%%%%%%%%%%%
\begin{frame}[allowframebreaks, noframenumbering,t]
\frametitle{Bibliography}
\nocite{*}
\printbibliography

\end{frame}

\end{document}